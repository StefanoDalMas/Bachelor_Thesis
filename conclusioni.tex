\chapter{Conclusioni e sviluppi futuri}
\label{cha:conclusioni}
In questo capitolo vengono presentate le conclusioni ed i possibili sviluppi futuri dell'applicativo.

\section{Conclusioni}
Sono state presentate le varie fasi dello sviluppo dell'applicativo DISI Challenge, un progetto che rende possibile ridurre la distanza tra la realtà dell'industria con quella universitaria, permettendo alle aziende di formalizzare Challenges ed agli studenti di formare dei gruppi per accedervi, tutto in un'unica WebApp.

Prima dello sviluppo dell'applicativo, è stata svolta un'accurata analisi delle tecnologie utilizzate in DISI Industry, per permettere di avere un'applicazione che lavorasse in sintonia con essa, in modo da poter avere un'esperienza utente unica, senza dover imparare ad utilizzare due applicativi diversi.

Per quanto concerne gli obbiettivi definiti, essi sono stati raggiunti in quanto l'applicazione permette le seguenti operazioni da parte dei vari attori:
\begin{itemize}
    \item Company Manager : creazione e gestione delle Challenges, oltre al fornire dei feedback sulla qualità del lavoro svolto dagli studenti.
    \item Studenti : partecipazione alle Challenges proposte dalle varie aziende, formando dei gruppi e gestendoli, oltre alla possibilità di creare dei feedback per le Challenges alle quali si è partecipati.
\end{itemize}

Oltre a questi attori, i principali all'interno dell'applicativo, sono stati definiti degli attori di supporto, tra i quali Moderatori e Tutor, figure che permettono di avere un controllo maggiore all'interno dell'applicativo.


\section{Validazione}
Ad ogni funzionalità aggiunta si sono svolti degli Unit Tests, ossia dei test volti al verificare che tale funzionalità sia implementata correttamente. Inoltre è stato necessario verificare che tutte le nuove funzionalità implementate non andassero a modificare il comportamento delle funzionalità già presenti, per questo motivo sono stati svolti anche degli Integration Tests.

Per quanto concerne la validazione dell'applicativo e per poter ottenere dei feedback ulteriori a quelli già ottenuti, questo progetto verrà presentato all'azienda \textbf{Hub Innovazione Trentino} \cite{HiT}. Tale azienda è stata scelta proprio per le sue caratteristiche, in quanto essa congiunge il mondo accademico con quello dell'industria permettendo di introdurre innovazione in quest'ultimo, dunque ricevere un feedback da un possibile futuro utilizzatore è sicuramente un'ottima occasione per poter migliorare l'applicativo.


\section{Sviluppi Futuri}
L'applicazione ha visto un ingrandimento durante lo sviluppo, ad esempio con l'aggiunta del concetto di \textbf{Reputazione} \ref{sec:valutazioni}, non presente inizialmente. Per permettere di sviluppare tutti i requisiti cercando di mantenere il miglior standard qualitativo, sono state omesse alcune funzionalità ritenute secondarie.

Valutando la soluzione nel complesso, è possibile ottimizzare alcune query effettuate con il database per permettere di avere un'applicazione più performante. Un altra componente che può essere migliorata è il front-end, in quanto non essendo questo progetto solamente adibito a ciò ma essendo un progetto di ingegneria del software, è stato dato più peso alla parte di back-end, lasciando la parte di front-end più basilare e volta alla dimostrazione delle funzionalità richieste più che al lato estetico, comunque gradevole data l'estensione dallo stile dell'applicativo orginiale.


La principale funzionalità che potrebbe essere implementata è la parte di comunicazione tra studente ed azienda mediante una Chat all'interno dell'applicazione. Inizialmente DISI Challenge non era pensata per tale scopo, solamente che dopo aver analizzato DISI Industry ed aver notato la presenza di una forma di comunicazione già presente, è stata aggiunta questa funzionalità, non implementata in quanto secondaria nei confronti del funzionamento dell'applicativo come definito in fase iniziale, ossia di portale per la pubblicazione di Challenges e la loro duale funzionalità di iscrizione ad esse. Sarebbe dunque interessante vedere l'implementazione di un sistema di comunicazione all'interno dell'applicativo sfruttando le tecnologie già presenti in Industry, e ciò dovrebbe essere relativamente semplice data la particolare attenzione posta per rendere DISI Challenge un modulo che lavora in sintonia con DISI Industry.

