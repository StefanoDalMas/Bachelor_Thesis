\chapter{Obbiettivi del progetto}
\label{cha:intro}
Il progetto ha come obbiettivo la realizzazione di una WebApp sottoforma di modulo da poter aggiungere alla piattaforma DISI Industry. L'applicativo deve permettere alle aziende, enti esterni all'università, di poter formalizzare e proporre delle "Challenge". Esse vengono definite come gare a lungo termine ove viene richiesto ai gruppi partecipanti di effettuare sia la fase di progettazione che eventualmente la fase implementativa di una soluzione rispetto ad un problema definito, oppure può essere proposta sottoforma di hackaton, ossia una competizione volte alla risoluzione di un problema in un lasso di tempo limitato mettendo in palio una serie di premi per i team vincitori.

Un esempio di Challenge che rientrano in questa categoria sono le Reply Challenge, hackaton nelle quali ai gruppi viene richiesto di risolvere dei problemi di algoritmica e di programmazione, oppure le Capture The Flag, hackaton nell'ambito della Cyber Security nel quale dei team concorrono contro l'organizzatore o tra di loro cercando di attaccare o difendere un sistema non sicuro

Questo progetto nasce dalla mia partecipazione al Samsung Innovation Camp, un evento proposto da Samsung in collaborazione con Randstad, nel quale veniva richiesto agli studenti di risolvere una problematica proposta dall'azienda, differenziandola sulla base dell'ateneo scelto, proponendo soluzioni che sfruttassero le nuove tecnologie quali Internet of Things, Intelligenza Artificiale e tutte le nuove tecnologie correlate. La problematica proposta all'ateneo di Trento era inerente alla valorizzazione del patrimonio naturalistico, artistico e culturale mediante l'innovazione digitale, con l'obbiettivo di sostenere il settore della Cultura e del Turismo.

La partecipazione a tale progetto mi ha permesso di conoscere nuove tecnologie, meccaniche aziendali e di lavorare in un team di sviluppo, permettendomi di migliorare le mie capacità di problem solving, di lavoro in team e di gestione del tempo, facendomi inoltre realizzare quanto sia importante avere una piattaforma nella quale gli studenti possano in modo semplice e veloce trovare tutte le informazioni necessarie per partecipare alle Challenge proposte dalle aziende.

Ho dunque deciso di implementare questa WebApp per permettere di ridurre lo spazio tra le aziende e l'università, fornendo dunque un portale nel quale questo tipo di interazioni siano facilmente accessibili agli studenti.



\section{DISI Industry}
\label{sec:context}
Durante la fase embrionale del progetto, ancora prima di iniziare a definire i requisiti, ho effettuato delle ricerche per vedere se esistesserò già soluzioni simili o comunque compatibili con la mia idea, ed ho trovato il progetto DISI Industry.


La scelta dell'integrazione di questo modulo all'interno di una WebApp già esistente è sorta dal fatto che DISI industry nasce già con l'idea di connettere gli studenti con l'industria, in quanto i primi possono cercare delle offerte di lavoro comparabili con le loro competenze apprese durante gli studi, mentre le compagnie possono offrire ruoli di lavoro per trovare il personale più adatto alle loro richieste. Essa può essere dunque definita come una piattaforma di recruiting online per gli studenti del DISI. \cite{industry}

Mi è dunque venuta naturale l'idea di implementare il modulo all'interno della WebApp per l'affinità intrinseca tra la mia ideologia e quella proposta dalla piattaforma già esistente, così da permettere agli studenti non solo di poter cercare potenziali offerte di tirocinio o di lavoro, ma anche di poter partecipare a delle Challenge proposte dalle aziende, così da poter mettere in pratica le proprie conoscenze e competenze apprese durante il percorso universitario, di poterle migliorare, oltre che di poter vincere dei premi, il tutto mediante la stessa piattaforma per ridurre al minimo la fatica dovuta alla navigazione di molteplici pagine web e portali. 



