\chapter*{Sommario} % senza numerazione
\label{sommario}

\addcontentsline{toc}{chapter}{Sommario} % da aggiungere comunque all'indice




  Sommario è un breve riassunto del lavoro svolto dove si descrive l'obiettivo, l'oggetto della tesi, le 
metodologie e le tecniche usate, i dati elaborati e la spiegazione delle conclusioni alle quali siete arrivati.  

Il sommario dell’elaborato consiste al massimo di 3 pagine e deve contenere le seguenti informazioni:
\begin{itemize}
  \item contesto e motivazioni 
  \item breve riassunto del problema affrontato
  \item tecniche utilizzate e/o sviluppate
  \item risultati raggiunti, sottolineando il contributo personale del laureando/a
\end{itemize}

Nel corso di questa tesi è stato affrontato lo sviluppo dell'applicativo DISI Challenge. In primo luogo è stata presentata l'idea iniziale, nata dalla mia partecipazione a delle Challenge promosse all'interno dell'università di Trento e dal mio desiderio di permettere agli studenti di poter parteciparvi in modo semplice ed intuitivo. 

Successivamente si è passati ad un'analisi delle tecnologie utilizzate da DISI Industry, piattaforma per la quale ho deciso di implementare ciò che è divenuto DISI Challenge, data 



