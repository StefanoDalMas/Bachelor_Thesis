\chapter*{Sommario} % senza numerazione
\label{sommario}

\addcontentsline{toc}{chapter}{Sommario} % da aggiungere comunque all'indice





Nel corso di questa tesi viene affrontato lo sviluppo dell'applicativo DISI Challenge, effettuato durante il mio tirocino interno all'università. Esso è una soluzione che permette alle compagnie di formalizzare e pubblicare in un portale delle Challenge, ossia delle sfide che possono variare per estensione temporale, ad esempio delle Capture The Flag o degli Hackaton di durata 1-2 giorni, oppure delle sfide a lunga durata dove viene richiesto di effettuare il design ed eventualmente l'implementazione di un progetto che risolva una task specifica fornita. Il vantaggio dell'esistenza di un tale applicativo è di ridurre al minimo i tempi di ricerca di questa tipologia di opportunità da parte degli studenti, permettendo loro di avere un' unica soluzione che possa in modo semplice ed intuitivo raccogliere tali sfide; mentre per le aziende è un'agevolazione nella divulgazione delle loro iniziative, con l'obbiettivo da una parte di trovare eventuali soluzioni al problema posto e dall'altra di avere la possibilità di individuare in modo efficiente ed efficace nuovi talenti da poter, al bisogno, introdurre nel loro contesto lavorativo.

Quando l'azienda pubblica una nuova Challenge all'interno del sistema è compito del moderatore, ossia la figura che gestisce uno o più laboratori all'interno dell'università, approvare o rifiutare una Challenge. Lo scopo di tale passaggio è quello di assicurarsi che le proposte presentate agli studenti siano di alta qualità. L'accettazione di una nuova Challenge da parte del moderatore comporta la disponibilità degli studenti di iscriversi ad essa, creando un gruppo o entrando a far parte di uno già esistente. Il secondo controllo viene effettuato dal Tutor, il quale rappresenta il supervisore del gruppo selezionato a partecipare ad una Challenge. Una volta che il Tutor ha accettato di entrare a far parte del gruppo, l'azienda ha a disposizione la possibilità di visualizzare i vari gruppi iscritti alla Challenge e di selezionare i gruppi più adatti alla sua Challenge. Il tutto è finalizzato ad avere un controllo di qualità sia da parte dell'azienda che da parte degli studenti. 


La mia idea è nata dopo aver partecipato ad una Challenge promossa all'interno dell'Università di Trento, ossia il progetto Samsung Innovation Camp, che mi ha portato a capire l'importanza dell'avere un unico portale che permetta a tutti gli studenti di poter visualizzare e partecipare alle iniziative proposte all'interno dell'Università da parte di aziende esterne. Il passo successivo è stato quello di proporre il mio progetto al Dott. Giorgini, il quale ha accettato l'idea e mi ha indirizzato verso il progetto DISI Industry, WebApp volta al permettere alle aziende di pubblicare offerte di lavoro o di tirocinio, così da ottenere il miglior candidato possibile per il ruolo da ricoprire, mentre lo studente ha a disposizione una piattaforma nella quale può trovare varie offerte e cercare quella più in linea con i suoi desideri e le sue competenze, il tutto mediante un unico applicativo.

La scelta di implementare la mia soluzione all'interno di DISI Industry è dovuta alla natura comune dei due progetti, ossia di avvicinare il mondo accademico a quello aziendale. L'apporto innovativo del mio progetto consiste nel permettere alle aziende di pubblicare delle Challenge ed agli studenti di parteciparvi con le finalità sopracitate. Il vantaggio derivato dall'unione dei due applicativi è quello di permettere agli utenti di poter usufruire di entrambi i servizi senza dover effettuare il login ad un altro portale e di poterlo fare direttamente da DISI Industry, in modo da avere una continuità di esperienza e di interfaccia.


Il lavoro successivo è stato quello di analizzare le tecnologie utilizzate da DISI Industry, cercando di comprendere appieno il funzionamento dei framework e dei sistemi adottati, per permettere di sviluppare una soluzione che fosse il più coerente e compatibile con la codebase già esistente, mantenendo dunque lo stile architetturale e grafico per non stravolgere l'esperienza dell'utente finale.


Dopo aver effettuato tale fase, sono passato a descrivere formalmente le varie componenti che costituiscono una Challenge. Tale processo è iniziato con la definizione degli attori del sistema e dei requisiti funzionali, ossia chi e cosa si può fare all'interno della WebApp e si è concluso con lo studio del processo per la creazione di una Challenge e la conseguente partecipazione degli studenti ad essa utilizzando DISI Challenge.

Un'ulteriore passaggio, dipendente dalla definizione degli attori e dallo studio delle tecnologie utilizzate, è stata l'analisi di contesto, necessaria per comprendere appieno come il modulo si interfacciasse agli strumenti già esistenti in DISI Industry ed agli attori definiti in precedenza, così da poter comprendere appieno come il modulo si inserisse all'interno del sistema già esistente, in modo da poterlo integrare senza stravolgere l'esperienza dell'utente finale.

La fase conclusiva è stata la modellazione del Back-End, che permettere la realizzazione di tali funzionalità e lo sviluppo del Front-End per realizzare un'esperienza gradevole e coerente al portale DISI Industry.

Durante lo sviluppo l'applicazione ha percepito un'ingrandimento del progetto iniziale con l'inserimento di un sistema di Feedback, che consiste in primo luogo di poter avere dei riscontri da parte delle aziende o di figure esperte nel settore sulla base della performance degli studenti. In secondo luogo a questi ultimi viene permesso di fornire delle risposte alle aziende, così da permettere a queste ultime di comprendere e migliorare le future Challenges che verranno proposte, ma anche per permettere ad altri studenti di avere un metro di paragone fornito dai loro colleghi durante la ricerca di nuove Challenges nel portale.

DISI Challenge è dunque risultato un progetto, che si è realizzato nello sviluppo di un modulo per la WebApp DISI Industry, che rispecchia i requisiti definiti inizialmente. Tra i possibili sviluppi futuri vi sono l'ottimizzazione e la rifinitura di alcune query poste al back-end, oltre ad apportare delle migliorie all'interfaccia front-end, la quale è stata sviluppata per permettere all'utente di utilizzare l'applicativo senza però porre troppa attenzione alle rifiniture. Tale scelta è stata adottata per permettere di proporre alla fine del lavoro un'applicazione leggermente meno rifinita ma completa di tutte le funzionalità richieste, piuttosto che rimuovere alcune di esse per prediligere la qualità dell'interfaccia grafica, che resta comunque elevata data la struttura iniziale di DISI Industry.

Ulteriormente nel futuro sarà possibile l'implementazione di una chat all'interno del modulo, che permetta alle aziende ed ai gruppi formati in occasioni di varie Challenges, di poter comunicare all'interno del portale senza dover occorrere a soluzioni e metodologie esterne, così da ridurre ulteriormente il carico da parte degli utenti finali e migliorarne l'esperienza.
